\documentclass{llncs} %
\bibliographystyle{splncs}
\usepackage{url}
\begin{document}

\title{Supporting Meditative Awareness through Neurofeedback Wearables}
\author{Vincenzo Pace}
\institute{Karlruhe Institute of Technology, Karlsruhe, Germany}
\maketitle
\newpage
\section{Introduction}
Adoption and functionality of wearable devices are increasing and expected to continue to do so \cite{Patel}. In the west, meditation is accepted more widely as a means of health improvement \cite{Tang:et al}.
Consumer EEG devices and freely available EEG/ERP databases \cite{ucsd} offer a chance to use algorithms to analyze a user's meditation experience and provide feedback for improvement.
This work presents the current state of art of consumer EEG devices, the signals involved in meditation and the possibility for usage of such devices for meditation training.
In the end, three popular EEG headsets will be compared and evaluated.
\subsection{EEG}
EEG stands for electroencephalography and is a process to record the electrical activity in the brain. The measured signal are voltage changes in and between neurons.
The brains signal is in the order of mikrovolt \cite{Berger}.  For measurement, EEG electrodes are placed on the test subjects scalp. In Research, up to 256 electrodes are used \cite{Seeck}, while consumer grade devices use around 10\% of this. \cite{Maskeliunas}
The electrical signal is measured as frequencies. These get passed to a Fast Fourier Transformation (FTT) or wavelet transformation \cite{Akin} to produce distinct waves, which are then categorized according to known patterns \cite{Shaker}.
Brain waves of individuals can be compared to aggravated data of many test subjects, to search for disorders \cite{Loo} or match for meditation success \cite{Tang:et al}, access to enough datapoints assumed.
There are four main brain frequencies \cite{Cahn}:
% quantitative eeg 
\begin{itemize}
    \item 
    Beta Waves (frequency range from 14 Hz to about 30 Hz)
    \item 
    Alpha Waves (frequency range from 7 Hz to 13 Hz)
    \item 
    Theta Waves (frequency range from 4 Hz to 7 Hz)
    \item 
    Delta Waves (frequency range up to 4 Hz)
\end{itemize}
From the data, conclusions about attention \cite{Berka}, stress \cite{Hosseini}, cognitive load \cite{Antonenko} and more are possible.
While EEG caps are usually used in research and academia \cite{Seeck}, headsets provide a reasonable solution for consumers with less complexity, but also lower precision\cite{Maskeliunas}.
For the accuracy of the measurement it is still crucial, that the sensors are correctly affixed \cite{Seeck}, so that they measure the brain regions that are involved in meditation.

EEG devices are either gel based or dry. In Research, gel electrodes are used and cost thousands of dollars, have wired signal transmission and are time consuming to setup and are uncomfortable to wear.
Consumer devices on the other hand are dry, wireless and cost only a few hundreds of dollars. \cite{Decho}
\subsection{Neurofeedback}
What is neurofeedback?

"Electroencephalography feedback has been used to aid training and study medita- tion by providing the practitioners with information on the brain waves they are producing"
\cite{Tang:et al}

How can it be used for the users advantage??
\subsection{Meditation}
Over the last decades, meditation has gained the interest of popular culture and the scientific community.
Documented beneficial effects range from mood improvements to structural changes \cite{Davidson} in various regions of the brain, such as the amygdala (responsible for stress and anxiety), anterior cingulate cortex, brain stem (breathing)
and the default mode network \cite{Tang:et al}.
Therefore, meditation has become part of psychiatric interventions \cite{Hoelzel}, high performance sports and personal development.
% QUOTE
Historically, meditation stems from buddhist practices and has been part of eastern culture for centuries, though variations were also practiced by the ancient greeks \cite{Hadot:Davidson}.

Though there is not one single meditation practice, the methods are plentiful. While often associated with stillness and watching ones breath,
there are also more active forms of meditation that involve more movement. Travis and Shear categorize meditation practices into focused attention, open monitoring and automatic self-transcending.\cite{Travis} 
Since this categorization is based on different EEG patterns and would be part of the wearable device, I will follow this categorization and differentiate where appropriate.

All these potential benefits lead to an increasing populace that wants to learn and practice meditation, so that a device that helps in the acquisition of this skill becomes desirable. 
The demands, functions, challenges and applications for such a device are the topic of this work.
\subsection{Awareness}
Mindfulness and awareness are two terms often mentioned in the context of meditation and there is no exact, agreed upon scientific definition to separate these two.
\section{Goals}
Since the beneficial effects of meditation are so various, the potential goals of using a wearable device 
that provides neurofeedback to its user are inherently diverse. Machine assisted meditation could help individuals to 
improve faster, track their progresssion, analyze the state of mind and lower the entrance barrier for novel practicioners. \cite{brand:del} \

The differences in EEG signals induced by the different meditation practices enable a categorization
and dedicated training regime to offer the device and software solution to a wider audience. \cite{Travis}
\section{Measurement Data}
\subsection{Biological Signals}
\begin{itemize}
    \item EEG
    \item brain blood flow infrared spectroscopy
\end{itemize}
Combination of NIRS and EEG would be optimal! There is no such device yet for consumers though.
Possible?
Brain wave explanation



Beta Waves (frequency range from 14 Hz to about 30 Hz)


Alpha Waves (frequency range from 7 Hz to 13 Hz)


Theta Waves (frequency range from 4 Hz to 7 Hz)


Delta Waves (frequency range up to 4 Hz)
\subsection{Measuring}
\subsection{Challenges}
Involuntary movement of the tongue, eyeballs, jaw and facial muscles produce noise in the milivolt range, while brain signals are on 
the order of mikrovolt. These artifacts do not only occur in consumer great EEG devices, but also in medical grade.
These artifacts need to be filtered out to produce a reliable signal for further analysis \cite{Bashivan: et al}.

The time window of measurement is also crucial for the accuracy of machine learning models.
\subsection{Processing and evaluation}
\cite{Lotte}
\section{Available Devices}
\subsection{Neurosky - Mindwave Mobile 2}
\subsection{InteraXon Inc. - Muse 2}
\subsection{Emotiv - Epoc+}
\section{Conclusion}


\begin{thebibliography}{[MT1]}
    \bibitem{brand:del}
    Tracy Brandmeyer, Arnaud Delorme,
    Meditation and neurofeedback
    Frontiers in Psychology, Article 688 (October 2013).
    \bibitem{Tang:et al}
    Tang, Yi-Yuan, Britta K. Hölzel, and Michael I. Posner. "The neuroscience of mindfulness meditation." Nature Reviews Neuroscience 16.4 (2015): 213-225.
    \bibitem{Bashivan: et al}
    Bashivan, Pouya, Irina Rish, and Steve Heisig. "Mental state recognition via wearable EEG." arXiv preprint arXiv:1602.00985 (2016).    
    \bibitem{Travis} 
    Travis, Fred, and Jonathan Shear. "Focused attention, open monitoring and automatic self-transcending: categories to organize meditations from Vedic, Buddhist and Chinese traditions." Consciousness and cognition 19.4 (2010): 1110-1118.
    \bibitem{Hoelzel}
    Hölzel, Britta K., et al. "How does mindfulness meditation work? Proposing mechanisms of action from a conceptual and neural perspective." Perspectives on psychological science 6.6 (2011): 537-559.
    \bibitem{Cahn}
    Cahn, B. Rael, and John Polich. "Meditation states and traits: EEG, ERP, and neuroimaging studies." Psychological bulletin 132.2 (2006): 180.
    \bibitem{Braboszcz}
    Braboszcz, Claire, and Arnaud Delorme. "Lost in thoughts: neural markers of low alertness during mind wandering." Neuroimage 54.4 (2011): 3040-3047.
    \bibitem{Zoefel} 
    Zoefel, Benedikt, René J. Huster, and Christoph S. Herrmann. "Neurofeedback training of the upper alpha frequency band in EEG improves cognitive performance." Neuroimage 54.2 (2011): 1427-1431.
    \bibitem{Berger}
    Berger, Hans. "Über das elektrenkephalogramm des menschen." European archives of psychiatry and clinical neuroscience 87.1 (1929): 527-570.
    \bibitem{Seeck}
    Seeck, Margitta, et al. "The standardized EEG electrode array of the IFCN." Clinical Neurophysiology 128.10 (2017): 2070-2077.
    \bibitem{Maskeliunas}
    Maskeliunas, Rytis, et al. "Consumer-grade EEG devices: are they usable for control tasks?." PeerJ 4 (2016): e1746.
    \bibitem{Surangsrirat}
    Surangsrirat, Decho, and Apichart Intarapanich. "Analysis of the meditation brainwave from consumer EEG device." SoutheastCon 2015. IEEE, 2015.
    \bibitem{Akin}
    Akin, Mehmet. "Comparison of wavelet transform and FFT methods in the analysis of EEG signals." Journal of medical systems 26.3 (2002): 241-247.
    \bibitem{Shaker}
    Shaker, Maan M. "EEG waves classifier using wavelet transform and Fourier transform." brain 2 (2006): 3.
    \bibitem{Lotte}
    Lotte, Fabien, et al. "A review of classification algorithms for EEG-based brain–computer interfaces." Journal of neural engineering 4.2 (2007): R1.
    \bibitem{Antonenko}
    Antonenko, Pavlo, et al. "Using electroencephalography to measure cognitive load." Educational Psychology Review 22.4 (2010): 425-438.
    \bibitem{Berka}
    Berka, Chris, et al. "Real-time analysis of EEG indexes of alertness, cognition, and memory acquired with a wireless EEG headset." International Journal of Human-Computer Interaction 17.2 (2004): 151-170.
    \bibitem{Hosseini}
    Hosseini, Seyyed Abed, and Mohammad Ali Khalilzadeh. "Emotional stress recognition system using EEG and psychophysiological signals: Using new labelling process of EEG signals in emotional stress state." 2010 international conference on biomedical engineering and computer science. IEEE, 2010.
    \bibitem{Loo}
    Loo SK, Barkley RA. Clinical utility of EEG in attention deficit hyperactivity disorder. Appl Neuropsychol 2005; 12(2): 64-76.
    \bibitem{Davidson}
    Davidson, Richard J., and Antoine Lutz. "Buddha's brain: Neuroplasticity and meditation [in the spotlight]." IEEE signal processing magazine 25.1 (2008): 176-174.
    \bibitem{Decho}
    Surangsrirat, Decho, and Apichart Intarapanich. "Analysis of the meditation brainwave from consumer EEG device." SoutheastCon 2015. IEEE, 2015.
    \bibitem{Hadot:Davidson}
    Hadot, Pierre; Arnold I. Davidson (1995) Philosophy as a Way of Life: Spiritual Exercises from Socrates to Foucault ISBN 0-631-18033-8 pages 83-84
    \bibitem{Patel}
    Patel, Mitesh S., David A. Asch, and Kevin G. Volpp. "Wearable devices as facilitators, not drivers, of health behavior change." Jama 313.5 (2015): 459-460.
    \bibitem{ucsd}
    \url{https://sccn.ucsd.edu/~arno/fam2data/publicly_available_EEG_data.html}



\end{thebibliography}
\end{document}